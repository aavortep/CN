\section*{РЕФЕРАТ}

Расчетно-пояснительная записка 36 с., 10 рис., 5 табл., 7 ист., 5 прил.

Объектом разработки в данной работе является база данных, содержащая информацию о репетиционных базах, соответствующих им комнатах и оборудовании, с целью предоставить возможность пользователям искать необходимые комнаты и бронировать свои репетиции. Цель данной работы – реализовать приложение, содержащее информацию о репетиционных базах. В приложении, работающем с этой БД, должна быть возможность для музыканта бронировать или отменять свои репетиции, а для владельца реп. базы - отслеживать записи на свою реп. базу.

Чтобы достигнуть поставленной цели, требуется решить следующие задачи:
\begin{itemize}
	\item формализовать задание, определить необходимый функционал;
	\item провести анализ СУБД;
	\item описать структуру БД;
	\item создать и заполнить БД;
	\item разработать ПО, которое позволит пользователю-музыканту бронировать и отменять свои репетиции, а владельцу отслеживать их;
	\item провести исследование зависимости времени выполнения запроса от числа записей в таблице.
\end{itemize}

Поставленная цель достигнута: в ходе курсового проекта была разработана база данных, хранящая информацию о репетиционных точках. При этом при разработке в качестве СУБД использовался PostgreSQL, а в качестве языка программирования – Python 3.7.

Дальнейшее развитие проекта подразумевает:
\begin{itemize}
	\item добавление фотографий комнат в блок информации о них;
	\item добавление календаря для более удобного бронирования репетиций;
	\item добавление возможности бронировать не только репетиционные базы, но и другие творческие площадки;
	\item добавление возможности администраторам блокировать пользователей;
	\item создание мобильной версии приложения.
\end{itemize}

КЛЮЧЕВЫЕ СЛОВА

\textit{базы данных, разработка ПО, репетиционные базы, бронирование репетиций, postgresql, python.}

\clearpage