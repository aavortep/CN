\section{Аналитическая часть}

\subsection{Анализ существующих решений}

На сегодняшний день существует множество различных торрент-клиентов. Ниже будут перечислены наиболее известные из них.
\begin{itemize}
	\item uTorrent
	
	uTorrent, также известный как µTorrent, является одним из наиболее широко используемых бесплатных торрент-клиентов. Однако на протяжении многих лет он вызывал некоторую критику из-за загруженности рекламой и дополнительным программным обеспечением.
	
	\item qBittorrent
	
	Этот торрент-клиент бесплатный и абсолютно лишённый рекламы в любом её виде. При этом в qBittorrent добавлены функции поиска торрентов по сайтам, возможности последовательной закачки и просмотра видео сразу после начала загрузки видео на компьютер.
	
	\item Vuze
	
	Популярный клиент в том числе среди пользователей Unix систем, старое название которого Azureus. В Vuze собран функционал по поиску, загрузке и перенаправлению на устройства, управлением скоростью для каждого отдельного файла, инструмент по конфигурации портов и множество других функций для профессиональной раздачи торрентов.
	
	\item BitTorrent
	
	Его можно назвать предком всех современных торрент-клиентов, так как это была первая программа, которая использовала протокол, на котором на сегодняшний день базируются все программы подобного типа. На данный момент этот клиент, хоть и не является лидером на рынке, всё равно считается самым надёжным и проверенным приложением среди всех конкурентов.
\end{itemize}

В целом, во всех перечисленных решениях можно выделить следующий общий функционал:

\begin{itemize}
	\item основа - протокол BitTorrent;
	\item поддержка файлов с расширением .torrent;
	\item поддержка функции загрузки данных как от сервера, так и от других клиентов;
	\item графический интерфейс с отображением состояния загрузки.
\end{itemize}

\begin{table}[!h]
	\captionsetup{justification=centering}
	\caption{\label{tab:data} Категории и сведения о данных}
	\begin{center}
		\begin{tabular}{|p{0.2\textwidth}|p{0.7\textwidth}|}
			\hline
			\textbf{Категория} & \textbf{Сведения}\\
			\hline
			Реп. база & Название, адрес, телефон, почта, кому принадлежит \\
			\hline
			Комната & Название, тип (вокал/группа и т. д.), площадь, стоимость за 3 часа, к какой репбазе относится \\
			\hline
			Оборудование в комнате & Тип (усилитель/ударные/микрофон и т. д.), бренд, количество, к какой комнате относится \\
			\hline
			Аккаунт & ФИО, телефон, почта \\
			\hline
			Репетиция & Время, какой музыкант (аккаунт), какая комната \\
			\hline
		\end{tabular}
	\end{center}
\end{table}

\newpage

\subsection{Принцип работы протокола BitTorrent}

\textbf{BitTorrent} -- это пиринговый (peer-to-peer, P2P) сетевой протокол для совместного обмена файлами через Интернет \cite{bittor_basics}.

Идея протокола заключалась в том, чтобы «разбить» передаваемый файл на более мелкие сегменты, называемые частями. Чтобы сэкономить пропускную способность, каждый фрагмент, который приобрёл скачивающий человек (\textbf{пир}), будет доступен для загрузки другим пирам в сети. Таким образом, большая часть нагрузки по обмену файлом ложится на пиры.

Также в рассматриваемом протоколе существует понятие \textbf{трекера}, необходимого для определения адресов пиров. Трекер хранит таблицу файлов и список пиров, имеющих данный файл в распоряжении.

Когда пир успешно получает все данные из содержимого торрента, этот пир становится \textbf{сидом}, то есть переходит в специальный режим работы, в котором он только отдаёт данные. Далее сид периодически информирует трекер об изменениях в состоянии торрентов и обновляет списки IP-адресов.

\subsection{Структура .torrent файла}

Торрент-файл содержит список файлов и метаданные целостности всех фрагментов, а также, при необходимости, содержит список трекеров. Он представляет собой словарь в кодировке Bencode со следующими ключами \cite{torrent_struct}:

\begin{itemize}
	\item announce -- URL трекера;
	\item info -- соответствует словарю со следующими ключами:
	\subitem files -- список словарей, каждый из которых соответствует файлу (если файлов несколько). Включает в себя length и path;
	\subitem length -- размер файла в байтах (при единственном файле);
	\subitem name -- имя файла (если файл один) или имя каталога (если файлов несколько);
	\subitem piece length -- количество байтов на фрагмент (обычно 256 КБ);
	\subitem pieces -- конкатенация SHA1-хешей каждого фрагмента (каждый хеш - 20 байт).
\end{itemize}

Все строки в .torrent файле, которые содержат текст, должны быть в кодировке UTF-8.

\subsection*{Выводы}

В этом разделе была проанализирована поставленная задача и уже существующие решения. А также было проанализировано устройство протокола BitTorrent, алгоритма взаимодействия с сервером и другими пирами.


